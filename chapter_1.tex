%%%%%%%%%%%%%%%%%%%%%%%%%%%%%%%%%%%%%%%%%%%%%%%%%%%%%%%%%%%%%%%%%%%%%%%%
%%%%%                                                              %%%%%
%%%%% Chapter 1: Introduction                                      %%%%%
%%%%%                                                              %%%%%
%%%%%%%%%%%%%%%%%%%%%%%%%%%%%%%%%%%%%%%%%%%%%%%%%%%%%%%%%%%%%%%%%%%%%%%%
\chapter{Introduction}
\label{ch1}

In the last decade, massively parallel high-throughput short-read
sequencing has revolutionized the efficiency and breadth of applications
for DNA sequencing \citep{kircher2010high}.  These high-throughput
sequencing methods produce millions to billions of short reads in a
single run, and have led to the development of many applications that
depend on ``read-counting'' to measure the abundance of specific
sequences in a sample. Examples include RNA-seq, ChIP-seq, and whole
genome copy number profiling.

Recently, long-read technologies have been developed that are filling
the gap left by short-read sequencers in applications such as genome
assembly \citep{jain2018nanopore,loman2015complete}, which benefit from
connecting more distant sequences within a contiguous molecule. Among
these the MinION instrument, from Oxford Nanopore Technologies, is
highly portable and inexpensive and has shown its unique value for
analysis outside of central sequencing facilities \citep{quick2016real}.
Long-read sequencers such as the MinION typically produce vastly fewer
reads from a sequencing run, and are therefore less efficient in
applications that use sequenced reads purely as a means to count
molecules. However, these technologies have the enormous advantage of
operating in near real-time, with a turnaround time that can be measured
in hours for some applications, rather than days or weeks.

Copy number variation (CNV) has been used successfully to understand a
variety of diseases \citep{sebat2007strong} -- notably cancers, which
exhibit both extreme variation and recurrent trends that can be used for
diagnostics and personalized approaches to treatment. For example, the
amplification and loss of certain genes, such as \textit{RB1} deletion and
\textit{MYCN} amplification in retinoblastoma, can be prognostic or even
predictive for treatment \citep{berry2017potential}.  High-throughput short-read
sequencing has been extremely effective in copy number profiling of
cancers \citep{chiang2009high}, including profiling single tumor cells
\citep{navin2011tumour}. However, for many potential users, the
efficiency of high-throughput short-read sequencing in CNV analysis is
determined by the availability of instruments and need for heavy
multiplexing to hit reasonable cost per profile. A sequencing core is
typically involved and an individual profile must wait for a ``full''
run before it can be processed. The MinION sequencer has an accessible
buy-in and is easy to use. Unfortunately the MinION has optimal
nucleotide throughput when producing reads that are orders of magnitude
longer than needed for CNV profiling.

To make full use of the advantages offered by the MinION sequencer, we
introduce sampling molecules using re-ligated fragments (SMURF)-seq, a
protocol to efficiently sequence short DNA molecules on a long-read
sequencer. The strategy of SMURF-seq is to concatenate short fragments
into very long molecules ($\sim$8 kb) prior to sequencing.
