%%%%%%%%%%%%%%%%%%%%%%%%%%%%%%%%%%%%%%%%%%%%%%%%%%%%%%%%%%%%%%%%%%%%%%%%
%%%%%                                                              %%%%%
%%%%% Chapter 1: Introduction                                      %%%%%
%%%%%                                                              %%%%%
%%%%%%%%%%%%%%%%%%%%%%%%%%%%%%%%%%%%%%%%%%%%%%%%%%%%%%%%%%%%%%%%%%%%%%%%
\chapter{Introduction}
\label{ch1}

In the last decade, massively parallel high-throughput short-read
sequencing has revolutionized the efficiency and breadth of applications
for DNA sequencing \citep{kircher2010high}.  These high-throughput
sequencing methods produce millions to billions of short reads in a
single run, and have led to the development of many applications that
depend on ``read-counting'' to measure the abundance of specific
sequences in a sample. Examples include RNA-seq, ChIP-seq, and whole
genome copy number profiling.

Recently, long-read technologies have been developed that are filling
the gap left by short-read sequencers in applications such as genome
assembly \citep{jain2018nanopore,loman2015complete}, which benefit from
connecting more distant sequences within a contiguous molecule. Among
these the MinION instrument, from Oxford Nanopore Technologies, is
highly portable and inexpensive and has shown its unique value for
analysis outside of central sequencing facilities \citep{quick2016real}.
Long-read sequencers such as the MinION typically produce vastly fewer
reads from a sequencing run, and are therefore less efficient in
applications that use sequenced reads purely as a means to count
molecules. However, these technologies have the enormous advantage of
operating in near real-time, with a turnaround time that can be measured
in hours for some applications, rather than days or weeks.

Copy number variation (CNV) has been used successfully to understand a
variety of diseases \citep{sebat2007strong} -- notably cancers, which
exhibit both extreme variation and recurrent trends that can be used for
diagnostics and personalized approaches to treatment. For example, the
amplification and loss of certain genes, such as \textit{RB1} deletion and
\textit{MYCN} amplification in retinoblastoma, can be prognostic or even
predictive for treatment \citep{berry2017potential}.  High-throughput short-read
sequencing has been extremely effective in copy number profiling of
cancers \citep{chiang2009high}, including profiling single tumor cells
\citep{navin2011tumour}. However, for many potential users, the
efficiency of high-throughput short-read sequencing in CNV analysis is
determined by the availability of instruments and need for heavy
multiplexing to hit reasonable cost per profile. A sequencing core is
typically involved and an individual profile must wait for a ``full''
run before it can be processed. The MinION sequencer has an accessible
buy-in and is easy to use. Unfortunately the MinION has optimal
nucleotide throughput when producing reads that are orders of magnitude
longer than needed for CNV profiling.

To make full use of the advantages offered by the MinION sequencer, we
introduce sampling molecules using re-ligated fragments (SMURF)-seq, a
protocol to efficiently sequence short DNA molecules on a long-read
sequencer. The strategy of SMURF-seq is to concatenate short fragments
into very long molecules ($\sim$8 kb) prior to sequencing. After (or
possibly concurrent with) sequencing, the SMURF-seq reads are mapped to
the reference genome by splitting them into their constituent fragments,
each aligning to a distinct location in the genome.
%
We demonstrate the utility of SMURF-seq with the low-cost MinION
sequencer to obtain data similar to that expected from typical
short-read sequencing, and generated high-quality copy number profiles
from this output.

More broadly, SMURF-seq is an approach for efficient short-read
sequencing, as required for read-counting, on long-read sequencing
machines. Here, we describe the details of the SMURF-seq approach; both
the SMURF-seq protocol prior to sequencing and mapping the sequenced
SMURF-seq reads. This study is organized as follows:

%%% Summary of chapters
% Chapter 2: Background
In the second chapter, we review the relevant background.  First, we
discuss the concept of nanopore sequencing and summarize its history,
the MinION sequencing instrument and its utility, library construction
methods, and properties of reads sequenced on these machines.
%
Then, we discuss the copy number profiling and its implications in
diversity and disease; especially its involvement in cancer, and the
utility of copy number analysis in understanding the biology of cancer
and diagnostic evaluation of tumors. We summarize protocols for copy
number profiling and computational methods to for generating profiles.
%
Finally, we discuss prior protocols that are similar in spirit to
SMURF-seq; these protocols include SAGE and its variants, SMASH, and
Concat-seq.

% Chapter 3: SMURF-seq
In the third chapter, we describe the details of the SMURF-seq approach
and demonstrate the accuracy of this approach for CNV profiling.  We
start with a discussion of sequencing long-reads or short-reads directly
for read-counting on nanopore machines, and the merits and limitations
of these methods.
%
Then, we introduce the SMURF-seq protocol for efficient short-read
sequencing on long-read machines; SMURF-seq combines the merits of
sequencing long or short reads directly, while alleviating the
limitations.
%
We demonstrate that SMURF-seq generates higher read-counts from a
sequencing run in comparison to these other method, the copy number
profiles generated with SMURF-seq are as accurate as profiles generated
on an Illumina platform, multiple samples can be multiplexed and
sequencing in the same sequencing run, and the reads generated in the
first few minutes of sequencing are sufficient to generate accurate
profiles.
%
Finally, we provide future directions for further improving the
efficiency and for expanding the utility of SMURF-seq.

% Chapter 4: Mapping SMURF-seq reads
The fourth chapter is dedicated to algorithmic and statistical aspects
of mapping SMURF-seq reads. We discuss the challenges associated with
mapping SMURF-seq reads as the fragments get shorter.
%
We introduce the fragment identification problem as a way of identifying
fragment boundaries and estimating the optimal number of fragments on a
SMURF-seq read.
%
Next, we define a score function for aligning SMURF-seq reads and
describe algorithms to find fragment boundaries on a read such that the
score is maximized.
%
Then, we determine the null distribution of aligning a SMURF-seq read
generated at random in order to calculate a p-value for a particular
fragmentation of a read. We use these p-values to determine the optimal
number of fragments on a read.
%
Finally, we suggest future directions of aligning SMURF-seq reads with
short fragments to large reference genomes.

% Chapter 5: Conclusion
We conclude this study by highlighting our vision of using the SMURF-seq
approach for short-read sequencing on long-read sequencers; we envision
that with further optimizations of SMURF-seq, to both the protocol and
mapping algorithms, would drive down the cost of sequencing and broaden
the applications of long-read sequencers.
