%%%%%%%%%%%%%%%%%%%%%%%%%%%%%%%%%%%%%%%%%%%%%%%%%%%%%%%%%%%%%%%%%%%%%%%%
%%%%%                                                              %%%%%
%%%%% Abstract                                                     %%%%%
%%%%%                                                              %%%%%
%%%%%%%%%%%%%%%%%%%%%%%%%%%%%%%%%%%%%%%%%%%%%%%%%%%%%%%%%%%%%%%%%%%%%%%%
\chapter*{Abstract}
% SMURF-seq
We present SMURF-seq, a protocol to efficiently sequence short DNA
molecules on a long-read sequencer by randomly ligating them to form
long molecules.
%
Applying SMURF-seq using the highly portable and inexpensive Oxford
Nanopore MinION yields up to 30 fragments per read, providing an average
of 6.2 and up to 7.5 million mappable fragments per run, increasing
information throughput for read-counting applications.

% Copy number
Somatic copy number alterations play a significant role in cancer, and
can be leveraged for diagnostic and personalized approaches to
treatment.
% Drawbacks of high-throughput short-read
High-throughput short-read sequencing has been extremely efficient in
copy number profiling; however, its applicability depends on the
availability of instrument, and time to obtain profiles can vary from a
few days to weeks.
% Copy number profiling on MinION
We apply SMURF-seq on the MinION to generate copy number profiles and
demonstrate that multiple samples can be multiplexed in a single
sequencing run.  A comparison with profiles from Illumina sequencing
reveals that SMURF-seq attains similar accuracy.

% aligning SMURF-seq reads
A SMURF-seq read is aligned to the reference genome by splitting it into
its constituent fragments, each aligning to a distinct location in the
genome.
% Fragment identification
We define a score function for aligning a SMURF-seq read and describe an
approach to determine the optimal fragmentation of a read.

% big picture
More broadly, SMURF-seq expands the utility of long-read sequencers for
efficient short-read sequencing, enabling applications on long-read
sequencers that are currently only efficient on high-throughput
short-read sequencers.
