%%%%%%%%%%%%%%%%%%%%%%%%%%%%%%%%%%%%%%%%%%%%%%%%%%%%%%%%%%%%%%%%%%%%%%%%
%%%%%                                                              %%%%%
%%%%% Abstract                                                     %%%%%
%%%%%                                                              %%%%%
%%%%%%%%%%%%%%%%%%%%%%%%%%%%%%%%%%%%%%%%%%%%%%%%%%%%%%%%%%%%%%%%%%%%%%%%
\chapter*{Abstract}
Somatic copy number alterations (CNA) play a significant role cancer,
and can be leveraged for diagnostic and personalized approaches to
treatment.
%
High-throughput short-read sequencing has been extremely efficient in copy
number profiling; however, its applicability depends on the availability of
instrument, and time to obtain profiles can vary from a few days to weeks.
%
We present SMURF-seq, a protocol to efficiently sequence short DNA molecules
on a long-read sequencer by randomly ligating them to form long molecules.
Applying SMURF-seq using the highly portable and inexpensive Oxford Nanopore
MinION yields up to 30 fragments per read, providing an average of 6.2 and up
to 7.5 million mappable fragments per run, increasing information throughput
for read-counting applications. We apply SMURF-seq on the MinION to generate
copy number profiles, demonstrate that multiple samples can be multiplexed and
sequenced in a single sequencing run, and show that concordant profiles are
obtained with reads sequenced in the first 45 minutes of a run. A comparison
with profiles from Illumina sequencing reveals that SMURF-seq attains similar
accuracy.
%% TODO: Need to write about the fragment identification problem
%
More broadly, with a fast and simple preparation method and a turnaround time
measured in hours, the SMURF-seq approach could provide a highly efficient
methodology for research and clinical laboratories where access to large-scale
sequencing is limited.
