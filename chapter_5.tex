%%%%%%%%%%%%%%%%%%%%%%%%%%%%%%%%%%%%%%%%%%%%%%%%%%%%%%%%%%%%%%%%%%%%%%%%
%%%%%                                                              %%%%%
%%%%% Chapter 5: Conclusions                                       %%%%%
%%%%%                                                              %%%%%
%%%%%%%%%%%%%%%%%%%%%%%%%%%%%%%%%%%%%%%%%%%%%%%%%%%%%%%%%%%%%%%%%%%%%%%%
\chapter{Conclusions}
\label{ch5}

Long-read sequencers, especially the Oxford Nanopore Technologies MinION
instrument, has widened the horizons of genome sequencing applications.
SMURF-seq pushes this boundary a little bit more by allowing such a
technology to be more efficiently leveraged in short-read sequencing for
read-counting applications, as required for copy number profiling.
%
The SMURF-seq approach sequences highly fragmented DNA molecules by
concatenating them into longer molecules, and thus, utilizing long-read
sequencers as optimized for long-read sequencing.

Copy number profiling as a diagnostic and prognostic tool to evaluate
cancer is expanding. With a fast and simple preparation method and a
turnaround time measured in hours, the SMURF-seq approach could provide
a highly efficient methodology for research and clinical laboratories
where access to large-scale sequencing is limited. Multiplexing samples
in a single run would drive down the cost of profiling significantly, and
thereby, we hope, help translate the research on clinical significance of
copy number profiling into patient outcomes.

Optimizing SMURF-seq for shorter fragments would enable generating
significantly higher read-counts from a single run of the MinION
instrument.  However, with shorter fragments, accuracy in identifying
fragment boundaries will begin to impact the ability of aligners to
recover fragments, and algorithms designed specifically to map SMURF-seq
reads will become essential.

We envision a broadening of the applications of SMURF-seq as the
underlying sequencing technology evolves and as SMURF-seq itself
improves by continual decrease in fragment lengths, increase in
sequenced read length, and data analysis methods optimized for SMURF-seq
resulting in an increase in information yield per nucleotide sequenced.
