%%%%%%%%%%%%%%%%%%%%%%%%%%%%%%%%%%%%%%%%%%%%%%%%%%%%%%%%%%%%%%%%%%%%%%%%
%%%%%                                                              %%%%%
%%%%% Appendix A: Detailed SMURF-seq protocol                      %%%%%
%%%%%                                                              %%%%%
%%%%%%%%%%%%%%%%%%%%%%%%%%%%%%%%%%%%%%%%%%%%%%%%%%%%%%%%%%%%%%%%%%%%%%%%
\chapter{Supplemental methods}
\label{appendA}

\section*{DNA samples}
The normal diploid female DNA was purchased from Promega (Cat. no.
G1521).  Breast cancer cell line SK-BR-3 (American Type of Culture
Collection (ATCC), Cat. no. HTB-30) was cultured in RPMI-1640 medium
(Thermo Fisher Scientific, Cat. no. 11875093) supplemented with 10\%
fetal bovine serum (FBS) (Thermo Fisher Scientific, Cat. no. 35011CV)
and was maintained at \SI{37}{\degree} in a humidified chamber supplied
with 5\% $\text{CO}_2$ and was regularly tested for mycoplasma
infection.

\section*{Cell lysis and DNA purification}
The DNA from SK-BR-3 cells was extracted and purified with the QIAamp
DNA Blood Mini Kit (Qiagen, Cat. no. 51104) following the protocol for
cultured cells given by the manufacturer. RNA and proteins in the cells
were degraded using RNase A stock solution
(\SI{100}{\milli\gram}/\SI{}{\milli\litre}) (Qiagen, Cat.  no. 19101)
and Protease-K (Qiagen, Cat. no. 19133) respectively. Both purchased
female diploid DNA and extracted SK-BR-3 DNA were treated with the same
downstream processes.

\section*{Fragmenting genomic DNA}
2-\SI{3}{\micro\gram} of genomic DNA was fragmented with restriction
enzyme Anza 64 SaqAI (Thermo Fisher Scientific, Cat. no. IVGN0644) for
\SI{30}{\minute} at \SI{37}{\degree}. The fragmented DNA was cleaned
with the QIAquick PCR purification kit (Qiagen, Cat. no. 8106) and
eluted with \SI{34}{\micro\litre} nuclease-free water. The concentration
of DNA was quantified on a Qubit Fluorometer v3 (Thermo Fisher
Scientific, cat. no. Q33216) with the Qubit dsDNA HS assay kit (Thermo
Fisher Scientific, cat. no. Q32854).

\section*{Ligation of fragmented DNA}
\SI{500}{\nano\gram} of fragmented DNA in \SI{10}{\micro\litre}
nuclease-free water was mixed with \SI{10}{\micro\litre} Anza T4 DNA
Ligase Master Mix (Thermo Fisher Scientific, Cat. no. IVGN210-4) and
incubated for \SI{30}{\minute} at room temperature. The ligated DNA was
cleaned with 2\(\times\) volume Ampure XP beads (Beckman Coulter, Cat.
no. A63881) and eluted in nuclease-free water. This step was done in
multiple tubes if more than \SI{500}{\nano\gram} of fragmented DNA was
needed to be ligated. The concentration of DNA was quantified on a Qubit
Fluorometer v3 with the Qubit dsDNA HS assay kit to ensure
\(\geq\)\SI{1}{\micro\gram} (\(\geq\) \SI{400}{\nano\gram}, if the Rapid
kit was used for library preparation) remained. The size of the ligated
DNA molecules were assessed with \(1\%\) agarose gel electrophoresis run
at \SI{90}{\volt} for \SI{30}{\minute}.

\section*{Library preparation (SQK-LSK108 1D DNA by ligation)}
\SI{1}{\micro\gram} of re-ligated DNA in \SI{45}{\micro\litre} of
nuclease-free water was end-repaired and dA-tailed (New England Biolabs
(NEB), Cat. no. E7546), followed by elution in nuclease-free water after
\(1.5\times\) volume Ampure XP beads clean-up. Sequencing adapters
(AMX1D) were ligated with Blunt/TA Ligase Master Mix (NEB, Cat.no.
M0367) and cleaned with \(0.4\times\) volume Ampure XP beads and eluted
using \SI{15}{\micro\litre} Elution Buffer (ELB) following the
manufacturer's protocol (Oxford Nanopore Technologies (ONT), 1D genomic
DNA by ligation protocol).

\section*{Multiplexed library preparation (EXP-NBD103 and SQK-LSK108)}
\SI{700}{\nano\gram} of each re-ligated sample in \SI{45}{\micro\litre}
of nuclease-free water was end-repaired, dA-tailed (NEB, Cat. no.
E7546), cleaned with \(1.5\times\) volume Ampure XP beads and eluted in
nuclease-free water. Different Native Barcodes (NB-x) for each sample
was ligated with Blunt/TA Ligase Master Mix (NEB, Cat.no. M0367),
cleaned with \(2\times\) volume Ampure XP beads and eluted in
nuclease-free water. Equimolar amounts of each sample was pooled to have
\SI{700}{\nano\gram} of DNA in \SI{50}{\micro\litre} water. Barcode
adapters (BAM) were ligated with Quick T4 DNA Ligase (NEB, Cat. no.
E6056), cleaned with \(0.4\times\) volume Ampure XP beads and eluted
using \SI{15}{\micro\litre} Elution Buffer (ELB) following the
manufacturer's protocol (ONT, 1D native barcoding genomic DNA).

\section*{Library preparation (SQK-RAD003 Rapid sequencing)}
\SI{400}{\nano\gram} of re-ligated DNA was concentrated with \(2\times\)
volume Ampure XP beads to \SI{7.5}{\micro\litre} nuclease-free water.
DNA was tagmented with Fragmentation Mix (FRA), and Rapid 1D Adapter
(RPD) was attached following the manufacturer's protocol (ONT, rapid
sequencing).

\section*{MinION sequencing and base-calling}
All the prepared libraries were loaded on R9.5 Flowcells following the
manufacturer's protocol (ONT) and sequenced for up to 48 hours using the
script specific to library preparation protocol. Base-calling and
de-multiplexing barcoded reads were performed using ONT Guppy (2.3.5)
with the appropriate parameters based on the library preparation kit.

\section*{Sequencing RE digested normal diploid genome}
% RE digestion
\SI{1}{\micro\gram} of genomic DNA was fragmented with restriction
enzyme Anza 64 SaqAI (Thermo Fisher Scientific, Cat. no. IVGN0644) for
\SI{30}{\minute} at \SI{37}{\degree}. The fragmented DNA was cleaned
with the QIAquick PCR purification kit (Qiagen, Cat. no. 8106) and
eluted with \SI{31}{\micro\litre} nuclease-free water. The concentration
of DNA was quantified on a Qubit Fluorometer v3 (Thermo Fisher
Scientific, cat. no. Q33216) with the Qubit dsDNA HS assay kit (Thermo
Fisher Scientific, cat. no. Q32854).

% 1D library prep.
\SI{0.5}{\micro\gram} of restriction enzyme digested DNA in
\SI{45}{\micro\litre} of nuclease-free water was end-repaired and
dA-tailed (New England Biolabs (NEB), Cat. no. E7546), followed by
elution in nuclease-free water after \(1.5\times\) volume Ampure XP
beads clean-up. Sequencing adapters (AMX1D) were ligated with Blunt/TA
Ligase Master Mix (NEB, Cat.no. M0367) and cleaned with \(1.0\times\)
volume Ampure XP beads (manufacturer's protocol uses \(0.4\times\)
volume XP beads, we increased to \(1.0\times\) to get as many short
molecules as possible) and eluted using \SI{15}{\micro\litre} Elution
Buffer (ELB) following the manufacturer's protocol (Oxford Nanopore
Technologies (ONT), 1D genomic DNA by ligation protocol).

% MinION sequencing and base-calleing
The prepared library was loaded on R9.4 Flowcell following the
manufacturer's protocol (ONT) and sequenced for 48 hours.  Base-calling
was performed using ONT Guppy (2.3.5).

\section*{Estimation of copy number variations}
CNV profiles were generated using the procedure described in
\citep{baslan2012genome,kendall2014computational} with the modification
employed in \citep{gerdtsson2018multiplex,malihi2018clonal}.  Briefly,
the human reference genome (hg19) was split into 5,000 (20,000 or
50,000) bins containing an equal number of uniquely mappable locations
and the bin counts were determined using uniquely mapped fragments.
Bins with spuriously high counts (`bad bins', typically around
centromeric and telomeric regions) were masked for downstream analysis
\citep{kendall2014computational}.  This procedure normalizes bin counts
for biases correlated with GC content by fitting a LOWESS curve to the
GC content by bin count, and subtracting the LOWESS estimate from each
bin \citep{kendall2014computational}.  Circular binary segmentation
(CBS) \citep{olshen2004circular}, implemented in DNAcopy
\citep{seshan2010dnacopy} package, then identifies breakpoints in the
normalized bin counts.  Following
\citep{gerdtsson2018multiplex,malihi2018clonal}, after CBS, spurious
segmentation calls were removed.  The influence of the GC content
correction can be seen in Additional file 1: Figure~S14.

\section*{Comparison with Illumina WGS of SK-BR-3 genome.}
DNA from SK-BR-3 cells was used to construct WGS library with the
NEBNext UltraII FS DNA Library Prep Kit (NEB, Cat. no. E7805) following
the manufacturer's instructions. After library quality and quantity
assessment with Qubit 3.0 HS dsDNA assay and BioAnalyzer HS dsDNA assay
(Agilent), libraries were sequenced on HiSeq 2500 (Illumina) with
single-end 130 cycles mode.

The reads were mapped with BWA-MEM using the default parameters, PCR
duplicates were removed, and CNV profiles were generated using exactly
the same method as used for SMURF-seq reads.  The scatter plots and
Pearson correlations comparing the CNV profiles were produced using R.

%%%%%%%%%%%%%%%%%%%%%%%%%%%%%%%%%%%%%%%%%%%%%%%%%%%%%%%%%%%%%%%%%%%%%%%%
%%%%%                                                              %%%%%
%%%%% Appendix B: Mapping SMURF-seq reads                          %%%%%
%%%%%                                                              %%%%%
%%%%%%%%%%%%%%%%%%%%%%%%%%%%%%%%%%%%%%%%%%%%%%%%%%%%%%%%%%%%%%%%%%%%%%%%
\chapter{Mapping SMURF-seq reads with long-read aligners}
\label{appendB}
\section*{Simulating SMURF-seq reads to evaluate mapping programs}
To test these mapping tools, we chose to create simulated reads with the
technical characteristics we expect in idealized SMURF-seq data. We
first selected a fragment length $\ell$ and a number $k$ of fragments
per read. Then, for a given WGS nanopore data set, we took the set of
mapped long reads as determined by BWA-MEM (with \texttt{-x ont2d}
option).
Each of the mapped reads was split into fragments of length $\ell$ (with
a random offset of $0$ to $\ell-1$ at the start of the long read). Each
fragment was validated by requiring that it did not overlap a deadzone
in the genome (as determined by the deadzone program available from
https://github.com/smithlabcode/utils for 40 bp). The reason for
excluding deadzones is that even when a short fragment has a ``known''
mapping location when it is part of a longer read, we cannot compare its
reported mapping location as a short fragment with that known location,
since we expect any good mapping algorithm to identify that the fragment
maps ambiguously. Among these validated fragments, subsets of $k$ were
sampled uniformly at random and concatenated (in random order and
orientation) to form simulated SMURF-seq reads.

The first and last fragments in a read should be slightly easier to
identify and map than the rest, since one of their boundaries is
known. Using the above procedure, we select $k=20$ so that the
simulated reads have a sufficient number of fragments to eliminate the
influence of the first and last fragments in each read on the
results. There is no need to have large $k$ otherwise.

By lowering $\ell$ and making the fragments shorter, the task of
mapping the fragments becomes more challenging. Real SMURF-seq reads
have fragment lengths determined by restriction site density, size
selection and other aspects of the experiments. But in testing mapping
algorithms and optimizing parameters, there is no disadvantage to
making the task more challenging. We only need to be able to
distinguish the relative performance of different mapping tools and
parameter combinations. Real SMURF-seq reads have varying fragment
lengths, but in evaluating mapping tools, there is no need to
randomize fragment lengths. None of the algorithms we evaluated are
capable of either deducing or leveraging the fact that all simulated
fragments have the same length. We selected $\ell = 100$, which
begins to challenge the various mapping strategies. These values of
$\ell$ are slightly lower than the average in real SMURF-seq data.


\section*{Evaluating performance using simulated SMURF-seq reads}
Within the simulated reads, the boundaries of each fragment are known
\textit{a priori}, as are their mapping locations. We used this
information to evaluate mapping tools in terms of (1) how well they
identify fragments purely for the purpose of counting molecules, which
is the primary information used in CNV analysis, and (2) how well they
identify individual mapping bases within reads. The latter criteria
becomes important in challenging cases and will be increasingly
important as fragment sizes are reduced.

Performance on identifying fragments: After mapping these simulated
reads, each mapping result is called a predicted fragment. Each
predicted fragment is considered a positive prediction, and we assume
an arbitrary order over positive predictions. A positive prediction is
a true positive if:
\begin{itemize}
\item The predicted fragment maps uniquely.
\item The mapping locations of at least half the bases in the
  predicted fragment are equal to the original mapping locations for
  those bases, and those bases are all part of the same original
  fragment (we assume that it is unlikely for two fragments on a simulated
  read to have the same mapping location but opposite orientation, and thus
  do not check for the orientation of a fragment). In this case, we say the
  predicted fragment is associated with that original fragment.
\item The predicted fragment is the first among predicted fragments
  associated the same original fragment.
\end{itemize}
False positives are predicted fragments that are not true positives. Any
original fragment with no associated predicted fragment is a false
negative. These criteria penalize splitting one original fragment or
merging two original fragments. By defining true positives, false
positives and false negatives we are able to calculate precision,
recall, and F-score for a particular mapping strategy.
%%%

Performance on identifying individual mapping bases: After mapping
simulated reads, each mapping result is decomposed into individual
nucleotides and associated with a location in the genome. Those
locations are retained. We keep multiplicities, so when two mapped
fragments overlap in the genome we count certain nucleotides twice.
These are the predicted positive bases in the reference.  The condition
positive bases are those known \textit{a priori} from the simulation.
The original fragment mapping locations may overlap in the reference
genome, leading to multiplicities in the condition positive bases, but
with low probability. The true positives are the intersection of the
condition positive and the predicted positive bases. When there are
multiplicities of mapped fragments and simulated fragments overlapping
the same bases in the reference genome, this is determined by taking the
smaller of the two values. After removing the true positives bases, the
remaining predicted positive bases are false positives, and the
remaining condition positive bases are false negatives. These criteria
penalize mapping approaches that do not cover the entire simulated
SMURF-seq reads, and also penalize approaches that predict fragments
that overlap within the read. The true positives, false positives, and
false negatives here allow us to assign precision and recall in terms of
individual bases and corresponding F-scores. Although the reference
bases for both predicted positive and condition positive could involve
multisets, since our simulations used relatively low coverage this
almost never happened.

To generate simulated reads we used the standard long reads from four
sequencing runs (Flowcell ID: FAB42704, FAB42810, FAB49914, and
FAF01253) in the public dataset available at \\
https://github.com/nanopore-wgs-consortium/NA12878/blob/master/Genome.md
\citep{jain2018nanopore,jain2018nanopore_git}. We downloaded the raw data
from EBI (Run accession: ERR2184696, ERR2184704, ERR2184712, and
ERR2184722) and base-called these with Guppy (version: 2.3.5).


\section*{Initial selection of mapping tools}
We tested the following mapping tools: BWA-MEM\citep{li2013aligning},
Minimap2\citep{li2018minimap2}, LAST\citep{kielbasa2011adaptive},
GraphMap\citep{sovic2016fast}, BLASR\citep{chaisson2012mapping},
rHAT\citep{liu2015rhat}, and LAMSA\citep{liu2017lamsa}. These were
selected either because they are known to perform well on certain
mapping tasks or have unique properties that plausibly could help in
mapping SMURF-seq reads. We tested each of these using default
parameters on simulated reads and downsampled real SMURF-seq
reads (data not shown). Among these BWA-MEM, Minimap2, and LAST had
higher accuracy on simulated data, and the other tools identified at
most 15 fragments per read on real data. Thus, we explored performance
of BWA-MEM (0.7.17), LAST (963), and Minimap2 (2.15) in more detail,
varying parameters to improve performance.

We remark that none of these tools were designed to map SMURF-seq reads;
results we report here do not reflect the overall performance of the
various mapping tools, only that the three aforementioned tools happened
to perform relatively well on a task for which they were not directly
designed for.


\section*{Determining the optimal Smith-Waterman score for
  SMURF-seq reads}
%% alignment score grid
In order to determine the optimal alignment score, we kept the seeding
related parameters constant, and varied the alignment score combinations
to perform a grid search. We varied the mismatch penalty from 1 to 6,
gap open penalty from 0 to 4, and gap extend penalty form 1 to 4.  The
match score was fixed at 1. Thus for each tool we tested 120 ($6 \times
5 \times 4$) combinations of alignment scores.

%% seeding parameter for each tool
The seeding and chaining related parameters for each tool was set at
follows (along with the four alignment scores):
\begin{itemize}
\item BWA-MEM: \texttt{-x ont2d -k 12 -W 12 -T 30}
\item Minimap2: \texttt{-w 1 -m 10 -s 30}
\item LAST (NEAR): \texttt{lastal -Q0 -e 20} and \texttt{last-split -m 1 -s 30}
\end{itemize}
We set the seeding and chaining parameters in a liberal manner to allow
for higher sensitivity than the default parameter of each tool, and
the minimum alignment score to output was set at 30.

%% selecting the best alignment parameter
After aligning the simulated reads, we calculated the average precision
and recall, each for the mapped fragment locations and nucleotides, for
the four datasets. The F-score was computed for each, and the mean of
the F-scores was used to determine the optimal alignment parameter for
each tool. Based on these results BWA-MEM outperformed other tools for
aligning SMURF-seq reads. BWA-MEM performed best with a mismatch, open,
and extension penalty of 2, 1, 1 respectively.

%% Zooming in on the grid for BWA
To further refine the optimal alignment parameter for BWA-MEM, we
aligned the simulated reads with parameter values around the value
described above with a higher resolution. We varied the mismatch penalty
from 1.5 to 2.5, and open and extend penalties from 0.5 to 1.5 in
increments of 0.25.
%
However, BWA-MEM does not accept floating point values for alignment
score parameters. To overcome this, we scaled the alignment score
proportionately to have integer values, i.e we varied the mismatch
penalty from 6 to 10, open and extend penalties from 2 to 6, and fixed
the match score at 4 (125 combinations).
%
Based on these results, the highest accuracy was obtained with the
mismatch, open, and extension penalty of 2.5, 1.5, 0.75 respectively
(corresponding scaled values are 10, 6 and 3). We used these optimal
alignment scores for mapping real SMURF-seq read, and all the CNV
profiles presented are based on these.



%%%%%%%%%%%%%%%%%%%%%%%%%%%%%%%%%%%%%%%%%%%%%%%%%%%%%%%%%%%%%%%%%%%%%%%%
%%%%%                                                              %%%%%
%%%%% Appendix C: Summary of sequencing runs                       %%%%%
%%%%%                                                              %%%%%
%%%%%%%%%%%%%%%%%%%%%%%%%%%%%%%%%%%%%%%%%%%%%%%%%%%%%%%%%%%%%%%%%%%%%%%%
\chapter{Data availability and summary of sequencing runs}
\label{appendC}

\begin{table}[H]
  \centering
  \begin{tabular*}{\columnwidth}{@{\extracolsep{\fill}}lrrrrr}
    \hline
    Sample & Kit & Reads & Mean length & Fragments & Accession \\
    % & & & length & & \\
    \hline
    Diploid & SQK-LSK108 & 270.82k & 6.8 kb & 7.28M & SRX5893474 \\
    Diploid & SQK-LSK108 & 497.92k & 3.7 kb & 7.55M & SRX5893475 \\
    SK-BR-3 & SQK-LSK108 & 146.98k & 7.6 kb & 4.52M & SRX5893478 \\
    SK-BR-3 & SQK-LSK108 & 132.64k & 7.3 kb & 4.02M & SRX5893479 \\
    \hline
    Diploid & SQK-RAD003 & 213.38k & 3.9 kb & 2.81M & SRX5893473 \\
    \hline
    Multiplexed run & EXP-NBD103 + & 442.9k & & &  \\
    Diploid (BC01) & SQK-LSK108 & 138.19k & 4.8 kb & 2.95M & SRX5893472 \\
    SK-BR-3 (BC02) &  & 144.57k & 7.7 kb & 4.97M & SRX5893476 \\
    \hline
    \hline
    Diploid (short-read) & SQK-LSK108 & 2.58M & 630.9 bp & & SRX5893480 \\
    \hline
    \hline
    SK-BR-3 (WGS) & Illumina WGS & 5.56M & 130 bp & &  SRX5893477 \\
    \hline
  \end{tabular*}
  \caption{Summary of sequencing run. Samples are processed with the
    SMURF-seq protocol, unless intdicated otherwise. Sequence data
    generated during the study are available in SRA with the accession
    number PRJNA454059.}
  \label{seq-summary}
\end{table}
