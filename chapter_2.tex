%%%%%%%%%%%%%%%%%%%%%%%%%%%%%%%%%%%%%%%%%%%%%%%%%%%%%%%%%%%%%%%%%%%%%%%%
%%%%%                                                              %%%%%
%%%%% Chapter 2: Background                                        %%%%%
%%%%%                                                              %%%%%
%%%%%%%%%%%%%%%%%%%%%%%%%%%%%%%%%%%%%%%%%%%%%%%%%%%%%%%%%%%%%%%%%%%%%%%%
\chapter{Background}
\label{ch2}

%%%%%%%%%%%%%%%%%%%%%%%%%%%%%%%%%%%%%%%%%%%%%%%%%%%%%%%%%%%%%%%%%%%%%%%%
%%%%%%%%%%%%%%%%%%%%%%%%%%%%%%%%%%%%%%%%%%%%%%%%%%%%%%%%%%%%%%%%%%%%%%%%
%%%%%%%%%%%%%%%%%%%%%%%%%%%%%%%%%%%%%%%%%%%%%%%%%%%%%%%%%%%%%%%%%%%%%%%%
\section{Nanopore sequencing}
%% Intro

%%%% History of nanopore development
%% Initial idea and expreiments
The concept of nanopore sequencing is based on the idea that as a
single-stranded DNA (or RNA) translocates through a nanopore (such as
transmemberane protein) in the presence of an electric field, the change
in current level would be proportional to the nucleotide passing through
the nanopore; Thus, measuring the current over time could be leveraged
to determine the sequence of nucleotides.
%
This idea of using transmemberane proteins for sensing and sequencing
nucleic acids was independently thought of by several researches
including David Deamer, Hagan Bayley, and George Church
\citep{deamer2016three,bayley2015nanopore,branton2010potential}.

%% Initial expreiments
% Detecting the presence of oligos
Initial experiments showed that as a single-stranded DNA or RNA
molecules could be driven through a \emph{Staphylococcus aureus}
$\alpha$-hemolysin in the presence of an electric field
\citep{kasianowicz1996characterization}. The current through the pore
remained constant in the absence of oligomers; The presence of oligomers
caused transient decreases in current, with the duration of the decrease
proportional to the length of the oligomer.
% Detecting the bases in oligos
Further research demonstrated that decrease in amplitude of current
could be used to different between poly-purine and poly-pyrimidine
sequencing of RNA \citep{akeson1999microsecond} and DNA
\citep{meller2000rapid}.

%% Challenges for sequencing DNA
Although these experiments demonstrated the potential for nanopores to
distinguish nucleic acid polymers, several challenges remained to be
addressed to use this approach for reading individual bases on a DNA or
RNA molecules.
%
The most important of these challenges were...


%% Oxford nanopore technologies

%% Current state of nanopore sequencing

%% Capture rate and fractors that affect it

%% Nanopore library construction
\paragraph{Library preparation for nanopore sequencing}

%% Oxford MinION

%% Properties of nanopore reads
\paragraph{Properties of nanopore reads}



%%%%%%%%%%%%%%%%%%%%%%%%%%%%%%%%%%%%%%%%%%%%%%%%%%%%%%%%%%%%%%%%%%%%%%%%
%%%%%%%%%%%%%%%%%%%%%%%%%%%%%%%%%%%%%%%%%%%%%%%%%%%%%%%%%%%%%%%%%%%%%%%%
%%%%%%%%%%%%%%%%%%%%%%%%%%%%%%%%%%%%%%%%%%%%%%%%%%%%%%%%%%%%%%%%%%%%%%%%
\section{Copy number variation and profiling}


%%%%%%%%%%%%%%%%%%%%%%%%%%%%%%%%%%%%%%%%%%%%%%%%%%%%%%%%%%%%%%%%%%%%%%%%
%%%%%%%%%%%%%%%%%%%%%%%%%%%%%%%%%%%%%%%%%%%%%%%%%%%%%%%%%%%%%%%%%%%%%%%%
%%%%%%%%%%%%%%%%%%%%%%%%%%%%%%%%%%%%%%%%%%%%%%%%%%%%%%%%%%%%%%%%%%%%%%%%
\section{Prior protocols based on concatenating DNA molecules}
% SAGE and its variants
The concept of ligating short DNA molecules to improve the efficiency
of sequencing was introduced in serial analysis of gene expression
(SAGE) \citep{}, and subsequently its variants such as LongSAGE and
SuperSAGE \citep{}. SAGE

% Digital karyotyping

% SMASH

% concat-seq

