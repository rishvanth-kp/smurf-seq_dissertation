%%%%%%%%%%%%%%%%%%%%%%%%%%%%%%%%%%%%%%%%%%%%%%%%%%%%%%%%%%%%%%%%%%%%%%%%
%%%%%                                                              %%%%%
%%%%% Chapter 4: Identifying SMURF-seq fragment boundaries         %%%%%
%%%%%                                                              %%%%%
%%%%%%%%%%%%%%%%%%%%%%%%%%%%%%%%%%%%%%%%%%%%%%%%%%%%%%%%%%%%%%%%%%%%%%%%

\chapter{Identifying fragment boundaries on a SMURF-seq read}
\label{ch4}

%%%%%%%%%%%%%%%%%%%%%%%%%%%%%%%%%%%%%%%%%%%%%%%%%%%%%%%%%%%%%%%%%%%%%%%%
%%%%%%%%%%%%%%%%%%%%%%%%%%%%%%%%%%%%%%%%%%%%%%%%%%%%%%%%%%%%%%%%%%%%%%%%
%%%%%%%%%%%%%%%%%%%%%%%%%%%%%%%%%%%%%%%%%%%%%%%%%%%%%%%%%%%%%%%%%%%%%%%%
\section{Motivation}


%%%%%%%%%%%%%%%%%%%%%%%%%%%%%%%%%%%%%%%%%%%%%%%%%%%%%%%%%%%%%%%%%%%%%%%%
%%%%%%%%%%%%%%%%%%%%%%%%%%%%%%%%%%%%%%%%%%%%%%%%%%%%%%%%%%%%%%%%%%%%%%%%
%%%%%%%%%%%%%%%%%%%%%%%%%%%%%%%%%%%%%%%%%%%%%%%%%%%%%%%%%%%%%%%%%%%%%%%%
\section{Background}
%% Similarities with the local alignment approach
A similar problem arose in the context of local alignment\cite{}.
Aligning any two sequences always gives an optimal alignment. However,
an approach was required to differentiate ``meaningful'' alignments from
alignment of unrelated sequences.
%%
The significance of the alignment score of sequences were determined
from a distribution of scores of aligning sequences (with the same
length and a similar base composition) generated at random.
%% Differences with the local alignment approach % We are interested in
%opt frags, and not significant alignment
However, the fragment identification problem differs from the local
alignment problem in a crutial manner. For the fragment identification
problem we have the reference genome, and it is assumed that the reads
always arise from this genome; the score distribution of sequences
genreated at random is used to determine the optimal number of
fragments, and not ``meaningful'' alignments.

In the early days of sequencing, with the advent of nucleotide datbases
\cite{} and algoritms for sequence alignment \cite{} an importane issue
was to determine significance of aligned sequences.
%% Very early studies

%% Simulation studies
In the context of local alignment, initial studies \cite{} analyzed the
distribution of alignment score of unrelated sequences with simulations.
These approaches assumed normaity of of distribution and determined
significance of a sequence from the number of standard deviations over
mean.

%% Mike's approach: coing tossing
% EVD approximation
% poisson approximation

%% Dembo and Karlin's approach

%% alignment allowing gaps

%% BLAST

%% Profile alignment

%% Global alignment score distribution

%% Differneces between these and the k=1 frag id problem




%% local alignment
% The local alignment score distribution has been studied extensively,
% especially in the context of BLAST score statistics
% \cite{altschul1990basic,altschul199627}.  The distribution of the local
% alignment score is well approximated by the extreme value distribution
% (EVD) \cite{karlin1990methods,
% karlin1990statistical,dembo1994limit,dembo1991strong}.

% It has also been shown that the distribution of the maximum score
% obtained when a profile sequence is aligned to all possible positions of
% a random sequence has a limiting extreme value distribution
% \cite{goldstein1994approximations}.


%%%%%%%%%%%%%%%%%%%%%%%%%%%%%%%%%%%%%%%%%%%%%%%%%%%%%%%%%%%%%%%%%%%%%%%%
%%%%%%%%%%%%%%%%%%%%%%%%%%%%%%%%%%%%%%%%%%%%%%%%%%%%%%%%%%%%%%%%%%%%%%%%
%%%%%%%%%%%%%%%%%%%%%%%%%%%%%%%%%%%%%%%%%%%%%%%%%%%%%%%%%%%%%%%%%%%%%%%%
\section{Fragment Identification problem}
%% string properties
Let $\Sigma$ be an alphabet. A string $X$ is a sequence of letters $a_0
a_1 \dots a_{n-1}$, where $a_i \in \Sigma$; $|X|$ denotes the length of
the string $X$; and $X[i \dots j] = a_i \dots a_{j-1}$ is a substring of
$X$.

%% reference string
The reference string $T$ is generated from the DNA alphabet $\Sigma =
\{A, T, G, C\}$, with $|T| = n$.
%% SMURF-seq string
A SMURF-seq read $S$ is generated by concatenating substrings (called
fragments) of $T$, with no information available \textit{a priori} about
the number, length, orientation (forward or reverse-complement), and the
position on $T$ of these fragments.  Further, $S$ contains sequencing
errors with a rate $\rho$. Let $|S| = m$ and $m \ll n$.

%% fragment set
A fragment set $P$ is an set of start locations of fragments on $S$. $P
\subset \{0 \dots m-1\}$ and $|P| = k$, with the rule that $0$ is in $P$
always.
%%
By convention we consider the set $P$ to be ordered such that if $i < j$
then $P_i < P_j$.
%%
For a fragment set $P$, $\sum_{i = 1}^{k} P_{i+1} - P_i = m$ and we say
that the $i^{\text{th}}$ of $S$ is the substring $S[P_i \dots P_{i+1}]$,
with $P_{k+1} = m$.

%% fragment identification problem
For a given $T$ and $S$, the fragment identification problem is to
determine the elements of the fragment set $P$ such that it corresponds
to the start locations of fragments contained in $S$.



%%%%%%%%%%%%%%%%%%%%%%%%%%%%%%%%%%%%%%%%%%%%%%%%%%%%%%%%%%%%%%%%%%%%%%%%
%%%%%%%%%%%%%%%%%%%%%%%%%%%%%%%%%%%%%%%%%%%%%%%%%%%%%%%%%%%%%%%%%%%%%%%%
%%%%%%%%%%%%%%%%%%%%%%%%%%%%%%%%%%%%%%%%%%%%%%%%%%%%%%%%%%%%%%%%%%%%%%%%
\section{Approach to the fragment identification problem}
%% Any k gives a score. Highest score does not correspond to the opt.
By the score function defined above, to determine the elements of the
fragment set $P$, requires the knowledge of the number of fragments $k$
and this is not known \textit{a priori}. Further, the $k$ that maximizes
the score function would almost never correspond to the optimal fragment
set. As an example, taking $k=m-1$ which corresponds to taking each base
as a fragment would maximize the score, however, this is a non-sensical
alignment.
%% So we need a way to detemine the optimal k.
Therefore, we require a approach to determine the optimal $k$.


We approach the fragment identification problem by defining a score
function as follows:
%% score function
For a given fragment set $P$, we define the score of aligning $S$ to $T$
as: \[score_T(S,P) = \sum_{i=1}^{k} \max\{score(T[u \dots v], S[P_i
\dots P_{i+1}]): 0 \leq u < v \leq n\}.\] This allows us to
consider the fragment identification problem as two inter-related
problems: (1) Determining $k$, the size of the fragment set, and (2)
given $k$, determining the elements of $P$ such that $score_T(S, P)$ is
maximized.

%% Brief description of approach


%%%%%%%%%%%%%%%%%%%%%%%%%%%%%%%%%%%%%%%%%%%%%%%%%%%%%%%%%%%%%%%%%%%%%%%%
%%%%%%%%%%%%%%%%%%%%%%%%%%%%%%%%%%%%%%%%%%%%%%%%%%%%%%%%%%%%%%%%%%%%%%%%
%%%%%%%%%%%%%%%%%%%%%%%%%%%%%%%%%%%%%%%%%%%%%%%%%%%%%%%%%%%%%%%%%%%%%%%%
\section{Score distribution under a random model}
\paragraph{Problem definition for a random model:}
The strings $T$ and $S$ are generated by drawing letters independently
from the same distribution from an alphabet $a \in \Sigma$ with
probability $p_a$ such that $\sum_{a \in \Sigma} p_a = 1$.
For a given fragment set $P$, what is the distribution of $score_T(S,
P)$?


\subsection{Score distribution of one fragment}
%% How fragment identification problem differs from these
The distribution of $score_T(S,1)$ differs from the local alignment as
we require an end-to-end alignment of $S$ to a substring of $T$. It also
differs from the profile score distribution since the letters of $S$ are
generated at random.

%% distribution for k = 1
Based on these results, the distribution of $score_T(S,1)$ is likely to
follow an extreme value distribution. The heuristic is as follows.
% heuristic of proof
Let $X_j$ denote the score of aligning $S$ with $T[j \dots j+m-1]$, then
\[X_j = \sum_{i=0}^{m-1} score(S[i],T[j+i]), j = 0, \dots, n-m+1.\]
Since the letters of T and S are iid, we have \[X_j \sim binom(m,p)\]
where \( p = \sum_{a=\Sigma} p_a^2\).  For a large enough $m$, $X_j$ can
be approximated by a normal distribution as \[X_j \sim N(mp, mp(1-p)).
\] $score_T(S,1)$ is the maximum score over all positions in $T$,
\[score_T(S,1) = \max_{0 \leq j \leq n-m+1} X_j.\] $score_T(S,1)$ is a
maximum of normal distributions, which follows an extreme value
distribution \cite{kotz2000extreme}.
%% m-dependence
Here, we have a dependence between $X_j$ and $X_k$ for $|j - k| < m$.


\subsection{Score distribution for a given fragment set}

\subsection{Score distribution for an unknown fragment set}


%%%%%%%%%%%%%%%%%%%%%%%%%%%%%%%%%%%%%%%%%%%%%%%%%%%%%%%%%%%%%%%%%%%%%%%%
%%%%%%%%%%%%%%%%%%%%%%%%%%%%%%%%%%%%%%%%%%%%%%%%%%%%%%%%%%%%%%%%%%%%%%%%
%%%%%%%%%%%%%%%%%%%%%%%%%%%%%%%%%%%%%%%%%%%%%%%%%%%%%%%%%%%%%%%%%%%%%%%%
\section{Estimating the optimal fragment set}


%%%%%%%%%%%%%%%%%%%%%%%%%%%%%%%%%%%%%%%%%%%%%%%%%%%%%%%%%%%%%%%%%%%%%%%%
%%%%%%%%%%%%%%%%%%%%%%%%%%%%%%%%%%%%%%%%%%%%%%%%%%%%%%%%%%%%%%%%%%%%%%%%
%%%%%%%%%%%%%%%%%%%%%%%%%%%%%%%%%%%%%%%%%%%%%%%%%%%%%%%%%%%%%%%%%%%%%%%%
\section{Identifying optimal fragment boundaries}


%%%%%%%%%%%%%%%%%%%%%%%%%%%%%%%%%%%%%%%%%%%%%%%%%%%%%%%%%%%%%%%%%%%%%%%%
%%%%%%%%%%%%%%%%%%%%%%%%%%%%%%%%%%%%%%%%%%%%%%%%%%%%%%%%%%%%%%%%%%%%%%%%
%%%%%%%%%%%%%%%%%%%%%%%%%%%%%%%%%%%%%%%%%%%%%%%%%%%%%%%%%%%%%%%%%%%%%%%%
\section{Results}

\subsection{Exact matching reads}

\subsection{Reads with mismatches}
